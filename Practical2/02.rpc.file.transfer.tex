\documentclass[a4paper,12pt]{article}
\usepackage[utf8]{inputenc}
\usepackage{graphicx}
\usepackage{listings}
\usepackage{xcolor}
\usepackage{float}
\usepackage{hyperref}

\lstset{
    language=Python,
    basicstyle=\ttfamily\small,
    keywordstyle=\color{blue},
    stringstyle=\color{red},
    commentstyle=\color{green!60!black},
    breaklines=true,
    frame=single,
    numbers=left
}

\title{Practical Work 2: RPC File Transfer}
\author{Name: [Dang Thu Huyen] \\ ID: [22BA13165] \\ Class: Cybersecurity - USTH}
\date{\today}

\begin{document}

\maketitle

\section{Introduction}
In this practical work, I upgraded the file transfer system from raw TCP sockets to \textbf{Remote Procedure Call (RPC)}. I utilized Python's built-in \texttt{xmlrpc} library, which implements the XML-RPC standard. [cite_start]This approach abstracts the networking details, allowing the client to invoke functions on the server directly[cite: 36, 37].

\section{RPC Service Design}
The RPC service is designed around a single remote procedure named \texttt{upload\_file}.
\begin{itemize}
    \item \textbf{Input:} Takes \texttt{filename} (string) and \texttt{binary\_data} (XML-RPC Binary object).
    \item \textbf{Process:} The server receives the binary object, extracts raw bytes, and writes them to the disk.
    \item \textbf{Output:} Returns a status string ("Success" or "Error").
\end{itemize}

[cite_start]Unlike the previous TCP implementation, there is no need to manually handle headers or handshake protocols, as the RPC middleware handles serialization (marshaling) automatically[cite: 30].

\begin{figure}[H]
    \centering
    \includegraphics[width=0.8\textwidth]{rpc_design.png}
    \caption{RPC Service Design (Function Call Flow)}
    \label{fig:design}
\end{figure}

\section{System Organization}
The system follows the standard RPC architecture described in the lecture (Slide 30).
\begin{itemize}
    \item \textbf{Client Side:} Uses a \texttt{ServerProxy} as a stub to marshal parameters (filename, file content) into XML format.
    \item \textbf{Server Side:} Uses \texttt{SimpleXMLRPCServer} to unmarshal requests and execute the registered function.
\end{itemize}

\begin{figure}[H]
    \centering
    \includegraphics[width=0.8\textwidth]{rpc_arch.png}
    \caption{RPC System Organization (Stubs and Skeleton)}
    \label{fig:arch}
\end{figure}

\section{Implementation}

\subsection{Server Implementation}
The server registers the function and listens for calls.
\begin{lstlisting}
# Code snippet: Server registering the function
def receive_file(filename, binary_data):
    with open(filename, "wb") as handle:
        handle.write(binary_data.data) # Extract raw bytes
    return "Success"

server.register_function(receive_file, "upload_file")
\end{lstlisting}

\subsection{Client Implementation}
[cite_start]The client wraps the file content in a Binary object to handle non-text data correctly[cite: 31].
\begin{lstlisting}
# Code snippet: Client calling the remote method
proxy = xmlrpc.client.ServerProxy("http://localhost:8000/")
with open("input.txt", "rb") as f:
    binary_data = xmlrpc.client.Binary(f.read())
    
proxy.upload_file("input.txt", binary_data)
\end{lstlisting}

\section{Conclusion}
The migration to RPC significantly reduced code complexity compared to raw sockets. [cite_start]The communication details (TCP connection, buffering) are successfully hidden by the RPC middleware[cite: 44].

\end{document}