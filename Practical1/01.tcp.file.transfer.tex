\documentclass[a4paper,12pt]{article}
\usepackage[utf8]{inputenc}
\usepackage{graphicx}
\usepackage{listings}
\usepackage{xcolor}
\usepackage{float}

\lstset{
    language=Python,
    basicstyle=\ttfamily\small,
    keywordstyle=\color{blue},
    stringstyle=\color{red},
    commentstyle=\color{green!60!black},
    breaklines=true,
    frame=single,
    numbers=left
}

\title{Practical Work 1: TCP File Transfer}
\author{Name: [Dang Thu Huyen] \\ ID: [22BA13165] \\ Class: Cybersecurity - USTH}
\date{\today}

\begin{document}

\maketitle

\section{Introduction}
The goal of this practical work is to implement a 1-1 file transfer system over TCP/IP using CLI. The system consists of a server (receiver) and a client (sender) utilizing socket programming.

\section{Protocol Design}
To ensure reliable file transfer, I designed a simple application-layer protocol:
\begin{enumerate}
    \item \textbf{Handshake:} The Client sends the filename and file size first (format: \texttt{filename|size}).
    \item \textbf{Acknowledgment:} The Server validates the metadata and sends back an "OK" signal.
    \item \textbf{Data Transmission:} The Client sends the raw file bytes in chunks (4096 bytes).
    \item \textbf{Termination:} The connection closes after the file size matches the received bytes.
\end{enumerate}

\begin{figure}[H]
    \centering
    \includegraphics[width=0.8\textwidth]{protocol_diagram.png}
    \caption{File Transfer Protocol Design}
    \label{fig:protocol}
\end{figure}

\section{System Organization}
The system architecture follows the Client-Server model described in the lecture (Slide 68):
\begin{itemize}
    \item \textbf{Server:} Creates a socket, binds to an IP/Port, listens for connections, and accepts the client.
    \item \textbf{Client:} Creates a socket and connects to the server's IP/Port.
\end{itemize}

\begin{figure}[H]
    \centering
    \includegraphics[width=0.8\textwidth]{system_arch.png}
    \caption{System Organization (Client-Server Session)}
    \label{fig:arch}
\end{figure}

\section{Implementation}
Below are the code snippets for the implementation in Python.

\subsection{Server Side}
\begin{lstlisting}
file_info = conn.recv(1024).decode('utf-8')
filename, filesize = file_info.split('|')
conn.sendall(b"OK")

with open(filename, 'wb') as f:
    while total_received < int(filesize):
        data = conn.recv(4096)
        if not data: break
        f.write(data)
\end{lstlisting}

\subsection{Client Side}
\begin{lstlisting}
client_socket.sendall(f"{filename}|{filesize}".encode())
ack = client_socket.recv(1024)

if ack == b"OK":
    with open(filename, 'rb') as f:
        while True:
            data = f.read(4096)
            if not data: break
            client_socket.sendall(data)
\end{lstlisting}

\section{Conclusion}
The system successfully transfers text and binary files over the local network. Error handling for connection loss and file existence was implemented.

\end{document}